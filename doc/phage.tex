\documentclass[12pt]{article}

\usepackage{mathrsfs}
\usepackage{url}
\usepackage{graphicx}   % need for figures
\usepackage{verbatim}   % useful for program listings
\usepackage{color}      % use if color is used in text
\usepackage{subfigure}  % use for side-by-side figures
\usepackage{fullpage}   % one-inch margins


%%% TITLE %%%
\title{phage: Topological Analysis of Phage Genome Relationships}
\author{Kevin Emmett and friends}
\date{\today}

\begin{document}

\maketitle

%%% ABSTRACT %%%
\begin{abstract}
Development of a consistent phage taxonomy is difficult.
\end{abstract}

%%% INTRODUCTION %%%
\section*{Introduction}
Phages are diverse.
Gene content analysis has been used previously to determine relationships.
Existing taxonomy established by the ICTV.
Genomically different phages can be classified together strictly based on morphological similarity.

%%% RESULTS %%%
\section*{Results}

%%% DISCUSSION %%%
\section*{Discussion}

%%% METHODS %%%
\section*{Methods}

\subsection*{Bacteriophage genomic data}
Bacteriophage gene profiles were taken from the POG dataset, compiled in \cite{Kristensen13}.
This dataset contained 1048 bacteriophage genomes and 4542 gene families.
Several genomic distance measures were used.
First, a similarity matrix was constructed as described in \cite{LimaMendez08}.

\subsection*{Persistent Homology}
The software package javaplex was used to compute persistence barcodes from the distance matrix described above.
Zero-dimensional homology generates a hierchical clustering of phage genomes that can be used for an initial taxonomic classification.
One-dimensional homology is reflective of loops within the dataset, indicating sets of genomes for which a strictly tree-like topology is inadequate.
A measure of topological obstruction can be defined.
\end{document}
